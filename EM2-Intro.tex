\section*{Presentaci�}

El material que es presenta a continuaci� s'ha originat en les
notes de classe de l'assignatura \emph{Estad�stica Matem�tica} que
hem impartit a la Diplomatura d'Estad�stica des del seu inici a la
Universitat de Barcelona.

L'objectiu d'aquests apunts no �s substituir els llibres citats a
la bibliografia sin�, m�s aviat, servir com a guia d'estudi per
tal que els estudiants puguin repassar els raonaments i els
c�lculs fets a classe i assegurar-se de que ho entenen tot
correctament.

Aquest document �s una versi� preliminar i, com a tal, pot
contenir algunes errates. Si ens hem animat a publicar-ho de forma
electr�nica, ha estat amb la idea que pugui resultar d'utilitat a
aquells a qui va destinat, no en un futur incert sin� des d'ara
mateix. Ens agradaria que ens f�ssiu arribar qualsevol errada,
errata o comentari.

\bigskip
Barcelona, 13 de febrer de 2002

\bigskip
�lex S�nchez Pla (alex@bio.ub.es) \\
Francesc Carmona (carmona@bio.ub.es)\\
Departament d'Estad�stica\\
Universitat de Barcelona

\thispagestyle{empty}
\newpage
