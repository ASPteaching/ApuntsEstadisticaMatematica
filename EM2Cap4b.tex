
\subsection{El m\`{e}tode del pivot en poblacions no normals}

Tots els exemples que hem vist fins ara suposen normalitat de la
poblaci�, per� el m�tode del pivot es pot aplicar sempre i quan
es donin les condicions amb que l'hem definit:
\begin{example}
\begin{eqnarray*}
X &\sim &f\left( x;\theta \right) =\frac{3x^{2}}{\theta
^{3}},\quad 0\leq x\leq \theta \\
L\left( x_{1}.......x_{n};\theta \right) &=&\prod_{i=1}^{n}\frac{3x_{i}^{2}}{%
\theta ^{3}}=\frac{3^{n}\prod_{i=1}^{n}x_{i}^{2}}{\theta ^{3n}}
\end{eqnarray*}

\begin{equation*}
\log L\left( \bX;\hat{\theta}\right) =n\log 3-2\sum \log
x_{i}-3n\log \theta
\end{equation*}

Observem que podem fer:
\begin{equation*}
\frac{\partial \log L\left( \bX;\hat{\theta}\right) }{\partial \log \hat{\theta%
}}=0\Rightarrow \frac{-3n}{\hat{\theta}}=0
\end{equation*}
i ens queda un expressi\'{o} que no t\'{e} sentit!!!!$\uparrow $.

Analitzem la forma de la funci\'{o} $\log L\left(
\bX;\hat{\theta}\right) $

\begin{equation*}
\log L\left( \bX;\hat{\theta}\right) \alpha =-2\sum \log
x_{i}-3n\log \theta =K(x)-3n\log \theta
\end{equation*}

Si $\theta <1$%
\begin{eqnarray*}
\log x_{i} &<&0\Rightarrow k\left( x\right) \\
-3n\log \theta &>&0
\end{eqnarray*}

\emph{L'EMV \'{e}s} $x_{n}\Leftarrow \left\{
\begin{array}{c}
\text{Si }\hat{\theta}^{\ast }<x_{\left( n\right) }\text{\emph{no
\'{e}s
versemblable}} \\
\text{Si }\hat{\theta}^{\ast }>x_{\left( n\right) }\text{ }L\left(
\bX;\theta \right) \text{\emph{\'{e}s m\'{e}s petit}}
\end{array}
\right\} $

Si
\begin{equation*}
X\sim f\left( \bX;\theta \right) =\frac{3x^{2}}{\theta ^{3}}\text{
\ \ \ \ \ \ \ \ \ \ }0\leq x\leq \theta
\end{equation*}
llavors, l'\emph{EMV} de $\theta $ \'{e}s $\hat{\theta}=x_{\left(
n\right) }$ .

La \emph{distribuci\'{o} de }$\hat{\theta}_{MV}$ \'{e}s:
\begin{equation*}
F\left( x_{\left( n\right) }\right) =P\left[ X_{\left( n\right) }\leq x%
\right] =P\left[ X_{\left( 1\right) }\leq x\right] \ast
.......\ast P\left[ X_{\left( n\right) }\leq x\right]
=\prod_{i=1}^{n}P\left[ X_{\left( i\right) }\leq x\right]
=P\left[ X\leq x\right] ^{n}
\end{equation*}

on
\begin{equation*}
P\left[ X\leq x\right] =\int_{0}^{x}\frac{3t^{2}}{\theta ^{3}}dt=\frac{3t^{3}%
}{3\theta ^{3}}\mid _{0}^{x}=\left( \frac{x}{\theta }\right) ^{3}
\end{equation*}
d'on
\begin{equation*}
F\left( x_{\left( n\right) }\right) =\left( \frac{x_{\left( n\right) }}{%
\theta }\right) ^{3n}=\left( \frac{\hat{\theta}}{\theta }\right)
^{3n}
\end{equation*}

Cosiderem la quantitat;
\begin{equation*}
T\left( x;\theta \right) =\left( \frac{x_{\left( n\right)
}}{\theta }\right) ^{3n}=F\left( x_{\left( n\right) }\right)
\end{equation*}

Aquesta funci\'{o} s'obt\'{e} aplicant la transformaci\'{o}
$F\Rightarrow $ \'{e}s una \emph{transformaci\'{o} integral }de
$x_{\left( n\right) }\Rightarrow \left\{
\begin{array}{c}
T\left( \bX;\theta \right) \sim U\left( 0,1\right) \\
F_{t}\left( t\right) =t \\
f_{t}\left( t\right) =1
\end{array}
\right\} $ No dep\`{e}n de $\theta \rightarrow $ Es una
\emph{quantitat pivotal}.

$T\left( \bX;\theta \right) =\left( \frac{x_{\left( n\right) }}{\theta }%
\right) ^{3n}\sim U\left( 0,1\right) $

Posem:
\begin{equation*}
P\left[ T_{1}(\alpha )\leq T(\bX;\theta )\leq T_{2}\left( \alpha \right) %
\right] =1-\alpha
\end{equation*}

i busquem els valors (dos funcions) que verifiquin aquesta
igualtat.

Si per simetria imposem que cada cua limiti una probabilitat
d'$\alpha /2$ tenim:
\begin{eqnarray*}
\int_{T_{1}\left( \alpha \right) }^{1}f_{T}\left( t\right) dt
&=&\int_{0}^{T_{2}\left( \alpha \right) }f\left( t\right) dt=1-\alpha \\
\int_{T_{1}\left( \alpha \right) }^{1}dt &=&\int_{0}^{T_{2}\left(
\alpha
\right) }dt=1-\alpha \\
1-T_{1}\left( \alpha \right) &=&1-\alpha \Rightarrow
1-T_{1}\left( \alpha
\right) =1-\alpha \\
T_{2}\left( \alpha \right) -0 &=&1-\alpha \Rightarrow T_{2}\left(
\alpha \right) =1-\alpha
\end{eqnarray*}

Ara posant
\begin{eqnarray*}
T\left( \bX;\theta \right) &=&\left( \frac{\bX_{\left( n\right) }}{\theta }%
\right) ^{3n}=T_{1}\left( \alpha \right) =\alpha \\
T\left( \bX;\theta \right) &=&\left( \frac{x_{\left( n\right) }}{\theta }%
\right) ^{3n}=T_{2}\left( \alpha \right) =1-\alpha
\end{eqnarray*}

Podem a\"{i}llar???????????????

\begin{eqnarray*}
T\left( \bX;\theta \right) &=&\left( \frac{x_{\left( n\right) }}{\theta }%
\right) ^{3n}=T_{1}\left( \alpha \right) =\alpha \\
\frac{x_{\left( n\right) }}{\theta } &=&\sqrt[3n]{\alpha } \\
????? &=&\theta \left( \bX\right) =\frac{x_{\left( n\right) }}{\sqrt[3n]{%
1-\alpha }}
\end{eqnarray*}

\begin{eqnarray*}
T\left( \bX;\theta \right) &=&\left( \frac{x_{\left( n\right) }}{\theta }%
\right) ^{3n}=T_{2}\left( \alpha \right) =1-\alpha \\
\frac{x_{\left( n\right) }}{\theta } &=&\sqrt[3n]{\alpha } \\
????? &=&\theta \left( \bX\right) =\frac{x_{\left( n\right) }}{\sqrt[3n]{%
1-\alpha }}
\end{eqnarray*}

Per disposar de dades amb que provar aquests intervals podem
generar una mostra amb $\theta =0.5$ de grandaria $n=9$. El
resultat de la simulaci� s�n els valors seg�ents:
\begin{equation*}
0.19,0.48,0.46,0.27,0.37,0.44,0.46,0.41,0.49
\end{equation*}

D'aqu\'{i} $x_{\left( q\right) }=$ maxim $=0.49$ i
l'\emph{intervals de confian�a} \'{e}s
\begin{equation*}
IC\Longrightarrow \left[ \frac{0.49}{\sqrt[27]{0.05}},\frac{0.49}{\sqrt[27]{%
0.95}}\right] =\left[ 0.5506,0.4937\right]
\end{equation*}
\end{example}

\subsection{Intervals de confian�a asimpt�tics}

Si la mostra es prou gran es possible aplicar el \emph{TCL
}b\'{e} sobre la mitjana $\sigma $ sobre $\hat{\theta}$ quan
aquest sigui \emph{AN}. El procediment es for\c{c}a general
perque l'\emph{EMV} es \emph{AN}.


\subsubsection{Intervals de confian�a en poblacions no normals}

Suposem que $X$ no segueix una \emph{D.\ Normal }per\`{o} si
t\'{e} mitjana i vari\`{a}n\c{c}a finites, i, que \emph{coneixem
}$\sigma .$

El \emph{TCL} ens garanteix que, en aquest cas
\begin{equation*}
\frac{\bar{X}-\mu }{\sigma /\sqrt{n}}\Rightarrow ^{\pounds
}N\left( 0,1\right)
\end{equation*}

i \ per tant, per mostres grans podem obtenir un \emph{intervals
de confian�a} per $\mu $ fins i
tot en el cas on $X$ no \'{e}s normal $\left( n\geq 30\right) $: Un intervals de confian�a al $%
\left( 1-\alpha \right) \%$ aproximat per a $\mu $ \'{e}s
\begin{equation*}
\hat{X}\pm Z_{\alpha /2}\sigma /\sqrt{n}
\end{equation*}

Suposem que $\sigma $\emph{\'{e}s desconeguda.} En aquest cas se
sol, tamb\'{e}, fer servir \emph{l'aproximaci\'{o} normal}, tot i
que, la seva validesa resulta molt discutible (Nom\'{e}s se sol
considerar v\`{a}lida quan $n\geq 100$).

Uns intervals de confian�a \emph{aproximadissims, }al $\left(
1-\alpha \right) \%$ per a $\mu $ s\'{o}n:

\begin{equation*}
\hat{X}\pm Z_{\alpha /2}\text{ }\hat{S}\text{ }/\sqrt{n}
\end{equation*}

Aquesta $\uparrow $ aproximaci\'{o} \'{e}s ampliamnet utilitzada.
